\newpage
\lecture{1}{February 12, 2021}{Introduction}
\section{Introduction: It's all connected}

%\subsection{Markov Equation}
The Markov equation is \[
	x^2 + y^2 + z^2 = 3xyz.
\]

Let's understand the integer solutions for the Markov equation.

\begin{defn}[Markov number]
	A Markov number $n \in \NN$ is any number such that there exists $y_0, z_0$ such that $(n, y_0, z_0)$ is a solution to the Markov equation.
	Let $m_n$ be the $n$-th positive integer Markov number.
\end{defn}

\begin{exmp}[Markov number]
	$(1, 2, 5)$ is a solution to the Markov equation. Thus,  $1, 2, 5$ are Markov numbers.
\end{exmp}

%\subsection{Diophantine Approximation}

\begin{thm}[Caracterization of irrational numbers]
	Let $\alpha \in \RR$. Then, $\alpha$ is irrational $\iff$ there are infinitely many coprime $(p, q)$ such that \[
		\left|\alpha - \frac{p}{q}\right| < \frac{1}{\sqrt{5}q^2}.
	\]
\end{thm}

\begin{thm}[$\sqrt{5}$ is the best constant]
	If $\alpha = \phi = \frac{1 + \sqrt{5}}{2}$ and $\beta > \sqrt{5}$, there are only finitely many coprime  $(p, q)$ such that \[
		\left|\alpha - \frac{p}{q}\right| < \frac{1}{\beta q^2}.	
	\]
\end{thm}

If we disregard $\phi$ and its derivatives, then we can change $\sqrt{5}$ to $2\sqrt{2}$.

\begin{thm}[Caracterization of irrational numbers not related to $\phi$]
	Let $\alpha \in \RR$. Then, $\alpha \not\in \QQ[\phi]$ $\iff$ there are infinitely many coprime $(p, q)$ such that \[
		\left|\alpha - \frac{p}{q}\right| < \frac{1}{2\sqrt{2}q^2}.
	\]
\end{thm}

\begin{thm}[$2\sqrt{2}$ is the best constant]
	If $\alpha = \sqrt{2}$ and $\beta > 2\sqrt{2}$, there are only finitely many coprime  $(p, q)$ such that \[
		\left|\alpha - \frac{p}{q}\right| < \frac{1}{\beta q^2}.	
	\]
\end{thm}

We can disregard $\sqrt{2}$ and its derivatives, and change $2\sqrt{2}$ to $\frac{\sqrt{221}}{5}$; and so on.

This naturally creates a sequence of real numbers, called \emph{Lagrange numbers}, which starts as $\sqrt{5}$, $2\sqrt{2}$, $\frac{\sqrt{221}}{5}$, $\dots$, $L_n$, $\dots$.

%\subsection{Connection \# 1}

Surprisingly, there is a conection between the Markov and Lagrange numbers.

\begin{thm}[Markov]
	\[
		L_n = \sqrt{9 - \frac{4}{m_n^2}}
	\]
\end{thm}

%\subsection{}

\begin{thm}
	Let $\rho_1, \rho_2 \in \Hom(F_2, SL(2, \RR))$. If  $\Tr(\rho_1(a)) = \Tr(\rho_2(a))$, $\Tr(\rho_1(b)) = \Tr(\rho_2(b))$ and $\Tr(\rho_1(ab^{-1})) = \Tr(\rho_2(ab^{-1}))$, then there exists $A \in SL(2, \RR)$ such that $\rho_1(w) = A \rho_2 A^{-1}$ for all $w \in F_2$.
\end{thm}

The upshot of this theorem is that $\Hom(F_2, SL(2, \RR)) / \text{conjugation}$ is, in some sense, a subset inside $\RR^3$.

For certain homomorphisms $\rho: F_2 \to SL(2, \RR)$, there exists a magical machine, which we will call \emph{hyperbolic geometry machine}, that sends $\rho$ to the following figure.

\begin{figure}[ht]
    \centering
	\incfig[.5]{hyperbolic-geometry-machine-of-certain-homomorphisms}
    \caption{Result of the hyperbolic geometry machine on certain homomorphisms}
    \label{fig:hyperbolic-geometry-machine-of-certain-homomorphisms}
\end{figure}

The length of the blue, green and red loops are replated to $\Tr(\rho(a))$, $\Tr(\rho(b))$ and $\Tr(\rho(ab^{-1}))$.

For certain super special homomorphisms  $\rho: F_2 \to SL(2, \RR)$, this machine sends $\rho$ to this other figure.

\begin{figure}[ht]
    \centering
	\incfig[.5]{hyperbolic-geometry-machine-of-super-special-homomorphisms}
    \caption{Result of the hyperbolic geometry machine on super special homomorphisms}
    \label{fig:hyperbolic-geometry-machine-of-super-special-homomorphisms}
\end{figure}

\begin{thm}
	$\rho$ is super special if, and only if, \[\Tr(\rho(a))^2 + \Tr(\rho(b))^2 + \Tr(\rho(ab^{-1}) = \Tr(\rho(a))\Tr(\rho(b))\Tr(\rho(ab^{-1})),\] i.e., $\left(\frac{\Tr(\rho(a))}{3}, \frac{\Tr(\rho(b))}{3}, \frac{\Tr(\rho(ab^{-1}))}{3}\right)$ is a solution to the Markov equation.
\end{thm}
