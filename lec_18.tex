\lecture{18}{March 31, 2021}{}

\begin{defn}[Differentiable complex functions in one variable]
	Let $f : \mathbb{C} \to \mathbb{C}$. Then, $f$ is called \emph{differentiable} or \emph{$\mathbb{C}$-differentiable} or \emph{holomorphic} if there exists a complex line through $(a, f(a)) \in \mathbb{C}^2$, given by some equation $y = f(a) + c(x-a)$ for some $c\in \mathbb{C}$ so that \[
		\lim_{x \to a} \frac{f(x) - \left( f(a) + c \cdot (x-a) \right)}{x-a} = 0,
	\]
	with $x \in \mathbb{C}$. If this occurs, then the \emph{derivative} of $f$ at $a$ is $c$.
\end{defn}

\begin{thm}[Cauchy-Riemann equations]
	Given $f : \mathbb{C} \to \mathbb{C}$, rewrite $f$ as \[
		f(x + iy) = u(x, y) + i\cdot v(x, y),
	\]
	for all $x, y \in \mathbb{R}$ and some $u, v: \mathbb{R}^2 \to \mathbb{R}$. 

	Then, $f$ is holomorphic at $z_0 = x_0 + iy_0 \in \mathbb{C}$ if, and only if, all of the following happen:
	\begin{enumerate}[label = (\textit{\roman*})]
		\item $u, v : \mathbb{R}^2 \to \mathbb{R}$ are differentiable at $(x_0, y_0)$.
		\item \[\frac{\partial u}{\partial x} (x_0, y_0) = \frac{\partial v}{\partial y} (x_0, y_0) \text{\quad and \quad}
			\frac{\partial u}{\partial y} (x_0, y_0) = -\frac{\partial v}{\partial x} (x_0, y_0).\]
	\end{enumerate}
\end{thm}
