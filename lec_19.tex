\lecture{19}{April 02, 2021}{}

\begin{figure}[htb]
    \centering
	\incfig[.8]{angles-are-preserved-in-holomorphic-functions.}
    \caption{Angles are preserved in holomorphic functions.}
    \label{fig:angles-holomorphic}
\end{figure}

\begin{prop}
	Suppose $f: \mathbb{C} \to \mathbb{C}$ is holomorphic. Rewrite $f$ as \[
		f(x+ iy) = u(x, y) + i\cdot v(x, y).
	\]
	Then, if we view $f$ as a function from $\mathbb{R}^2 \to \mathbb{R}^2$, via \[
		f(x, y) = (u(x, y), v(x, y)),
	\] its Jacobian matrix $\Jac f$ has orthogonal columns. So, $\Jac f$ is an orthogonal matrix. In especial, if $\vec a, \vec b \in \mathbb{R}^2$, then $\angle(\vec a, \vec b) = \angle(\Jac f \cdot \vec a, \Jac f \cdot \vec b)$.
\end{prop}

Therefore, holomorphic functions are approximated (to better and better accuracy) by orthogonal linear transformations. So long as $f' \neq 0$, this means that angles between curves are preserved, as seen in \cref{fig:angles-holomorphic}.
